% @brief    LaTeX2e Resume for James Crooks
\documentclass[margin,line]{resume}

\begin{document}
\name{\Large James E. Crooks}
\begin{resume}

    % Contact Information
    \section{\mysidestyle Contact\\Information}
    826 E Hyde Park Blvd, Unit 3    
    \hfill cell: (727) 215-0447 
    \vspace{0mm} \\
    \vspace{0mm}
    Chicago, IL, 60615
    \hfill e-mail: crooks1379@gmail.com
    \vspace{0mm} \\
    \vspace{0mm}
    \hfill github: https://github.com/emptylist 
    \vspace{0mm} \\
    \vspace{-4.5mm}
    
    % Education
    \section{\mysidestyle Education}

    \textbf{University of Chicago}, Chicago, IL \vspace{2mm} \\
    \vspace{1mm}
    \textsl{Ph.D. Biophysical Sciences} \hfill \textbf{ 2010 -- present} \vspace{-3mm}\\\vspace{-1mm}%
    \\
     Expected graduation date: July 2015 \\
     I study the molecular dynamics of the T cell receptor using molecular dynamics simulations.  I construct Markov models from the simulated data to build multiscale, discrete models of the dynamics to better understand the solution state dynamics of the system and how the flexibility of the CDR loops contributes to antigen recognition.  In the process of my research, I have studied and built software systems to manage terabyte-scale data sets, perform non-linear dimensionality reduction (manifold learning) of large in-memory data sets, and to automate the process of building markov models from data.  Additionally, I have contributed computational expertise in molecular docking and automation procedures to two other lab papers, and developed a molecular dynamics protocol and wrote data analysis scripts for a third paper outside my personal research.
    \vspace{-1mm}

    \textbf{University of Florida}, Gainesville, FL \vspace{2mm} \\
    \vspace{1mm}
    \textsl{B.S. Mathematics (cum laude)}, Physics Minor \hfill \textbf{2010}\vspace{-3mm}\\\vspace{-1mm}%
    \vspace{-2mm}

    \section{\mysidestyle Work Experience}

    \textbf{Artifice, NFP} \\
    \textsl{Co-founder \& CTO} \hfill \textbf{Oct 2013 -- present} \vspace{-3mm}

    Artifice, NFP (www.artificechicago.org) is a direct service not-for-profit corporation located in the Woodlawn neighborhood on the south side of Chicago.  Artifice is a technology education center and youth hackerspace aimed at teaching underpriviledged youth computer programming, web design and development, robotics, and entrepreneurial skills.  As a volunteer instructor, primarily teaching our introductory bootcamp (covering basic HTML5 and CSS3), Arduino and Python programming.
    As CTO, I oversee technology backing the company, and act as lead developer on the Relic attendance tracking project.  Relic is an RFID-based system for tracking attendance at youth programs in Chicago for which Artifice was awarded a grant from Hive Chicago to develop.  I have written the embedded networking, scheduling, and encryption firmware for the RFID-devices.  Server-side development is currently underway.
    \vspace{-2mm}

    \textbf{HERE, a Nokia Company} \\
    \textsl{R\&D Intern} \hfill \textbf{June 2014 -- September 2014} \vspace{-3mm}

    I studied the problem of determining when legal speed limits have been changed based on probe data collected from consumers via phones and connected cars.  A solution to the problem was valued at \$8M/yr by the company.  I applied machine learning methods translated from the scientific literature on robotics to develop a prototype solution that successfully solved the problem on test data sets extracted from Illinois data for which a speed limit ground truth was known, but was strongly confounded by weather events.
    
    \vspace{-2mm}

    % Honours and Awards
    \section{\mysidestyle Select Honors and Awards} 
    NSF Graduate Research Fellowship, University of Chicago, 2012-15 \vspace{1mm} \\
    NIH Training Grant, University of Chicago, 2010-12 \vspace{1mm}\\
    HHMI Extramural Research Fellowship, University of Florida, 2009  \vspace{1mm}\\
    \vspace{-8mm}\\

    % Publications
    \section{\mysidestyle Scientific Publications}
    GT Tietjen, Z Gong, C-H Chen, E Vargas, \textbf{JE Crooks}, KD Cao, CTR Heffern, JM Henderson, M Meron, B Lin, B Roux, ML Schlossman, TL Steck, KYC Lee, EJ Adams,
    ``A molecular mechanism for differential recognition of membrane phosphatidylserine by the immune regulator receptor Tim4'',
    \textsl{PNAS}, doi:10.1073/pnas.1320174111
    \vspace{-3mm}

    A Sandstrom, C-M Peigne, A Leger, \textbf{JE Crooks}, R Breathnac, M-C Gesnel, M Bonneville, E Scotet, EJ Adams,
    ``The B30.2 domain of Butyrophilin 3A1 binds phosphoantigens to mediate activation of human V$\gamma$9V$\delta$2 T cells'',
    \textsl{Immunity}, doi:10.1016/j.immuni.2014.03.003
    \vspace{-3mm}

    \pagebreak

    J Lopez-Sagaseta, C Dulberger, \textbf{JE Crooks}, AM Luoma, A McFedries, I van Rijhn, A Saghatelian, EJ Adams, 
    \textbf{2013}
    ``The molecular basis for MAIT cell recognition of MR1'',
    \textsl{PNAS} doi:10.1073/pnas.1222678110

    \vspace{-2mm}

    % Presentations
    \section{\mysidestyle Industry Presentations}
    \emph{Artifice: building an educational ARG with the (physical) cloud}, Cloud Camp Chicago, Jan 23, 2014 (with Ashley Lane)
    \vspace{-2mm}
    % Computer Skills
    \section{\mysidestyle Programming and Computer Skills} 

    Python, NumPy/SciPy, JavaScript, Haskell, C, C++, Haskell, Bash Shell scripting, Linux, Cray cluster use, Machine learning (particularly manifold learning), statistical data analysis, and \LaTeX \\
    \vspace{-2mm}

\end{resume}

\end{document}

% EOF

